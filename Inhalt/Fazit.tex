% !TEX root = ../Ausarbeitung.tex
\newpage
\section{Fazit und Ausblick} 
\label{sec:Fazit}
% Aktuelle Lage
In den letzten Jahren stieg die Nutzung von Containern rasant an und viele Unternehmen investieren massiv in diese Technologie (vgl. \Abschnitt{AktuelleLage}). 
Die ausschlaggebenden Gründe hierfür sind der verringerte Overhead, die Performance-Vorteile und das vereinfachte Deployment von Anwendungen. 

% Softwareentwicklung
Im Bereich der Softwareentwicklung bringen Container einen großen Vorteil. 
Es ist möglich innerhalb von Sekunden ein System mit den nötigen Eigenschaften aufzusetzen, um die Anwendung darauf auszuführen. 
Außerdem kann so sichergestellt werden, dass Test- und Produktivsystem absolut identisch sind und keine unerwarteten Effekte auftreten. 

% Business
Durch die immer simpleren Containertechnologien können Unternehmen mit wenig Budget und Personal die Containerisierung ihres Unternehmens vorantreiben und die Vorteile dieser ausschöpfen. 
Somit ist auch in den nächsten Jahren mit dem vermehrten Einsatz von Containertechnologien im Businessbereich zu rechnen. 

% Container und VMs
Allerdings haben Container auch Schwächen, wie z.B. die vorhandenen Abhängigkeiten vom Betriebssystem.
Daher sind sie kein Ersatz für virtuelle Maschinen. Vielmehr ist es ratsam, die beiden Technologien in Kombination miteinander zu verwenden. 
Dabei ist es vor allem zu empfehlen den Docker-Host zu virtualisieren, sodass dieser bei einem Hardwareausfall trotzdem weiter laufen kann und bei Wartungen der Dienst weiter zur Verfügung steht. 
Somit können die in \Abschnitt{Einleitung} genannten Vorteile genutzt werden.

% Container Technologien
Durch die Diversität bei den Containertechnologien können die Anwender Container auf den verschiedensten Betriebssystemen nutzen.
Zudem wird durch die Konkurrenz der Technologien untereinander die Weiterentwicklung und Optimierung der Technik gefördert. 
Auch wurde die \ac{OCI} gegründet, um die Technologie zu standardisieren und eine gemeinsame Weiterentwicklung zu fördern.

