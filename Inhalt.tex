% !TEX root = Ausarbeitung.tex
%\input{Inhalt/Einleitung}
%\input{Inhalt/Funktionalität}
%\input{Inhalt/Containertechnologien}
%\input{Inhalt/ContainerUndSD}
%\input{Inhalt/Cluster}
%\input{Inhalt/Risiko}
%\input{Inhalt/AktuelleLage}
%\input{Inhalt/HS}
%\input{Inhalt/Fazit}


\section{Einleitung}
\lipsum[2]
\subsection{Projektbeschreibung}
\lipsum[2]
\section{Annahmen}
\lipsum[2]
\section{Projektinhalt}
\lipsum[2]
\subsection{Aktivitäten}
\lipsum[2]
\subsection{Work-Breakdown-Strukture}
\lipsum[2]
\subsection{Oranisation Breakdown Strukture}
\lipsum[2]
\section{Zeitmanagement}
\lipsum[2]
\subsection{Aktivitätendauer}
\lipsum[2]
\subsection{Meilensteine}
\lipsum[2]
\subsection{PERT} % Kritischer Pfad !
\lipsum[2]
\subsection{Gant} % Kritischer Pfad !
\lipsum[2]


\section{Kommunikationsplan}
\lipsum[2]
\subsection{Stakeholder}
\lipsum[2]
\subsection{Regelmeetings} % Wer, wo, wann, wie oft, Medien?
\lipsum[2]
\subsection{Statusberichte} % Wo gesammelt und verteilt
\lipsum[2]
\subsubsection{Template für Statusberichte}
\lipsum[2]
\section{Qualität}
\lipsum[2]
\subsection{Qualitätsprozesse} % Code Review, UnitTest
\lipsum[2]
\subsubsection{High und low-Leveldesign}
\lipsum[2]
\subsubsection{Code Review}
\lipsum[2]
\subsubsection{Unit Test}
\lipsum[2]
\subsubsection{Test}
\lipsum[2]
\subsubsection{Review}
\lipsum[2]
\subsection{Ticketsytem}
Als Ticktsystem kommt das firmeneigene Jira zum Einsatz. Dieses ist über die Adresse \texttt{https://jira.doofenschmirtz.evil} verfügbar. Alle Entwickler und Product Owner besitzen ein Zugang zu diesem System.
\subsection{Versionierungssystem}

Als Versionierungssystem für das Projekt wird Git eingesetzt. Dieses ist allen Entwicklern auf ihren Computern verfügbar. Als Git-Remote dient der firmeneigene Bitbucket-Server, der unter der Adresse \texttt{git.doofenschmirtz.evil} verfügbar ist. Auch aus dem Internet ist der Server unter dieser Adresse verfügbar.

Eine Commit-Message muss immer die getätigte Arbeit beschreiben und eine eindeutige Zuordung zu einem Ticket oder ein User Story ermöglichen. Dazu werden diese über ihre eindeutige Bezeichnung (US-3, BUG-5) erwähnt.

\subsection{Codestyle} % A-Style
\lipsum[2]
\subsection{Bugverlauf}
\lipsum[2]
\section{Risikoplan}
\lipsum[2]
\subsection{Annahmen} % Auswahl
\lipsum[2]
\subsection{Risiken mit Priorität} %3 Stück
\lipsum[2]
\subsection{Bewertung } %Wahrscheinlichkeitsanalyse, Risikomatrix
\lipsum[2]
\subsection{Risikokosten}
\lipsum[2]
\section{Human Ressources}
\lipsum[2]
\subsection{Aufgabenverteilung}
\lipsum[2]
\subsection{Aktionsplan}
\lipsum[2]
\subsection{Motivation} % Teamevents, Awards .. (kostet Geld)
\lipsum[2]

\section{Beschaffung} % UserStories an extenern Firma

\section{Kostenmanagement}
\lipsum[2]
\subsection{Kostenschätzung}
\lipsum[2]
\subsection{Personalkosten} % Awards, Reise, Hotel, Betriebssätze ...
\lipsum[2]
\subsection{Materialkosten}
\lipsum[2]
\subsection{Risikokosten}
\lipsum[2]
\subsection{Gewinnmarge}
\lipsum[2]
\subsection{Kontigenz} %
\lipsum[2]
\subsection{Plankosten}
\lipsum[2]





